\documentclass{article}
\usepackage{graphicx}
\usepackage{dsfont}

\begin{document}

\title{Team 4 - Visibility Checks}
\author{Wang, Mao, Sebastian G.}

\maketitle

\section{Problem Description}
Given
\begin{itemize}
  \setlength{\itemsep}{-2pt}
  \item A graph $G=(V,E)$ describing a landscape with obstacles modeled through simple polygons with $n$ vertices in total
  \item A straight-line drawing of the landscape $\Gamma : V \rightarrow \mathds{R}^2$
  \item The position $P \in \mathds{R}^2$ (\textit{Pacman})
  \item The positions $(Q_i \in \mathds{R}^2)_{i \in \{1, ..., m\}}$ of $m$ ghosts
  \item The visibility radius $r$
\end{itemize}
Find an efficient way to determine if $P$ is visible from $Q_i$ for each
$i \in \{1, ..., m\}$, using ``reasonable'' preprocessing time
(e.g. about a minute on typical inputs).
\par\bigskip

For positions $A, B \in \mathds{R}^2$, $A$ is \textit{visible} from $B$ iff
$|\overline{AB}| \leq r$ and the segment $\overline{AB}$ has no intersection with any polygon in $\Gamma$.

\section{Suggested Approach}

We took inspiration in ray tracing and settled for a solution involving a spatial
data structure like a BVH tree, quadtree or kd-tree. This is to bring down the
asymptotic complexity of visibility tests to $\mathcal{O}(m\log n)$, but comes
at the cost of having to build such that data structure at preprocessing time, which
will happen in $\Omega(n\log n)$.
\par\bigskip

The outline of the algorithm is as follows:

\begin{enumerate}
  \setlength{\itemsep}{-2pt}
  \item Triangulate the input polygons (or split them in any other kind of convex polygon) for efficient intersection tests
  \item Build the spatial data structure on this (convex) polygon soup
  \item Perform the $m$ visibility checks through intersection tests with the data structure in $\mathcal{O}(\log n)$ each
\end{enumerate}


\end{document}
